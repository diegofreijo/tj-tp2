% Archivo de configuracion del informe
% -------------------------------------------

\usepackage[spanish,activeacute]{babel}								% Idioma castellano
\usepackage{caratula}														% Caratula de Algo2
%\usepackage[a4paper=true,pagebackref=true]{hyperref}				% Agrega la TOC al PDF e hipervinculos
\usepackage{graphicx} 														% Permite insertar graficos
\usepackage{fancyhdr}														% Permite manejo de cabeceras de pagina
\usepackage{eufrak}															% Usado en el enunciado del trabajo
\usepackage{latexsym}
\usepackage{algorithmic}													% Para escribir los algos
%\usepackage{dsfont}															% Para el simbolo de naturales
\usepackage[font=small,labelfont=bf]{caption}						% Para editar las captions


\usepackage[utf8]{inputenc}
\usepackage{amsmath}
\usepackage[x11names, rgb]{xcolor}
%\usepackage{tikz}
%	\usetikzlibrary{snakes,arrows,shapes}
\usepackage{listings}
\usepackage{lastpage}
\usepackage{geometry}
  \geometry{left=1cm, right=1cm, top=2cm, bottom=2cm}

% Estilo de pagina para tener las cabeceras y pieseras
\pagestyle{fancy}
  \fancyhead[LO]{Teor\'ia de Juegos}
  \fancyhead[C]{Trabajo Práctico}
  \fancyhead[RO]{Primer Cuatrimestre 2008}
  \renewcommand{\headrulewidth}{0.4pt}

  \fancyfoot[LO]{Diego Freijo - Maximiliano Giusto}
  \fancyfoot[C]{}
  \fancyfoot[RO]{P\'agina \thepage\ de \pageref{LastPage}}
  \renewcommand{\footrulewidth}{0.4pt}

\parindent = 1.5 em 
\parskip = 8 pt


%%%%%%%%%%%%%%%% COMANDOS %%%%%%%%%%%%%%%%%%%%
\newcommand{\todo}{{\large\textbf{TODO: }}}
\newcommand{\paso}{\textsc{Paso }}
\newcommand{\func}[1]{\verb|#1|}
\newcommand{\imagen}[3]
{
	\begin{figure}[htbp]
	  \centering
	    \includegraphics[scale=#3]{../img/#1}
	  \caption{#2}
	\end{figure}
}

\newcommand{\imagenVertical}[3]
{
	\begin{figure}[htbp]
	  \centering
	    \includegraphics[angle=90, scale=#3]{../img/#1}
	  \caption{#2}
	\end{figure}
}

\newcommand{\nat}{\mathds{N}}
\newcommand{\algoritmosc}[2]{\vspace{2em} \noindent {\bf\underline{#1}:} #2}
\newcommand{\algoritmo}[3]{\algoritmosc{#1}{#2} $\longrightarrow$ #3}
\newcommand{\superindice}[1]{$^\textrm{{\tiny #1}}$}
\newcommand{\subsubsubsection}[1]{\noindent\negrita{#1}

}
\newcommand{\negrita}[1]{{\bf #1}}
\newcommand{\italica}[1]{{\it #1}}

%%%%%%%%%%%%%%%% FIN COMANDOS %%%%%%%%%%%%%%%%%%%%
